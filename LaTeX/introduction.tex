Climate change represents one of the most critical challenges facing humanity today, influencing global 
policy agendas, stimulating widespread activism, and generating intense public debate \cite{10.1093/oxfordhb/9780199566600.003.0001}. As the world 
faces with the implications of rising temperatures, melting ice caps, and extreme weather events, 
understanding the public discourse around climate change is more crucial than ever. The conversations 
and narratives that unfold on public platforms significantly shape the societal response and 
policy-making towards this existential threat. Therefore, studying these discussions can provide 
invaluable insights into public sentiment and the evolution of the climate change debate.
This thesis explores the linguistic properties on climate change using a dataset from Reddit, a 
popular online platform where millions of users engage in discussions covering a variety of topics. 
The dataset, sourced from Kaggle and titled "The Reddit Climate Change Dataset", includes posts 
and comments from January 2010 to the end of August 2022. It offers a rich corpus for examining how 
conversations about climate change have evolved over a significant period, marked by crucial
international agreements \cite{unfccc2015paris}, scientific advancements \cite{DUSONCHET2015986,RUBIN2015378}, 
and shifts in global climate policy \cite{unep2020emissiongapreport}.
The overarching research question guiding this study is: "What are the patterns of public sentiment 
and discourse on climate change in social media, and how do they evolve over time?" To address this, 
the thesis is structured around two specific research questions:

\begin{enumerate}
    \item How does the frequency of specific unigrams and bigrams about climate change evolve over time?
    \item How does the sentiment associated with specific unigrams and bigrams about climate change evolve over time?
\end{enumerate}

The primary aim of this thesis is to conduct a detailed analysis of the climate change discussions on 
Reddit, providing an overall overview of the discourse and examining specific topics to track their 
development through the years. By using natural language processing (NLP) techniques, this study 
focuses on identifying major key terms and topics, understanding the context in which they appear, and observing 
how they change over time. This approach not only sheds light on the shifting priorities within the 
climate change conversation but also highlights the varied perspectives and ideological divides that 
characterize public opinion on this issue.
A central component of this research involves performing sentiment analysis on the dataset. Sentiment 
analysis, a sub-field of NLP, involves computationally identifying and categorizing opinions expressed 
in text, especially to determine whether the discourse is positive, negative, or neutral. By applying 
sentiment analysis to Reddit comments about climate change, this thesis aims to capture the 
emotional tone of public sentiment. This analysis will employ VADER (Valence Aware 
Dictionary and sEntiment Reasoner), a lexicon and rule-based sentiment analysis tool that is 
specifically attuned to sentiments expressed in social media.
Additionally, the thesis reports a frequency distribution analysis to quantify the most commonly discussed 
topics and terms within the dataset, providing quantitative backing to the qualitative insights by offering 
concrete, measurable evidence of how often certain topics and terms are discussed, which complements and 
strengthens the subjective interpretations of the data. This will enable a structured understanding of which 
aspects of climate change are most engaging or disputed among Reddit users. Moreover, Named Entity Recognition 
(NER) will be used to identify and categorize key entities such as persons, locations, organizations, and 
nationalities mentioned in the discussions. This will aid in understanding the geopolitical and social dimensions 
of the climate change debate as expressed by the global Reddit community.

Nonetheless, nowadays the availability of automated systems and machine learning technologies allows for a 
matchless exploration of such large datasets. These tools not only improve the efficiency and 
accuracy of our analyses but also offer new possibilities for discovering insights that were previously 
inaccessible through manual methods alone. This capability is particularly significant in environmental 
studies, where the rapid assessment of public opinion and discourse patterns can inform timely and 
effective policy responses.
The general relevance of this research extends beyond academic interests, touching upon practical 
implications for policymakers, environmental organizations, and climate activists. By understanding 
the dynamics of online discussions, stakeholders can better strategize their communications, target 
their interventions, and engage with the public in more meaningful ways. Moreover, this thesis 
contributes to the broader field of digital humanities by demonstrating how data from social media 
platforms can be used to gain insights into societal issues.
In conclusion, this bachelor thesis not only seeks to provide a comprehensive analysis of the climate 
change discourse on Reddit but also aims to contribute to our understanding of how digital platforms 
influence and reflect public opinion on critical global issues. By combining computational 
techniques and qualitative assessments, the study aims to offer a detailed picture of the 
changing landscape of climate change discussions, making a significant contribution to both academic 
research and practical applications in environmental communication and policy-making.

\vspace{1em}

All the materials and scripts used in this thesis can be found here:
\url{https://github.com/nurzatdzholchubekova/BachelorsThesisNDzholchubekovaLMU2024}