\subsubsection{English}
This thesis explores patterns of public sentiment and discourse on climate change in social media by analyzing Reddit data from 2010 to 2022 using Natural Language Processing (NLP) techniques. The study investigates how the frequency and sentiment of specific terms evolve over time. Findings reveal a significant increase in climate change discussions, particularly during events like the Fridays for Future protests. Sentiment analysis shows a predominantly negative tone overall, but many years had more positive sentiment, reflecting optimism about climate solutions. Recent years, however, were marked by increased negative sentiment, indicating rising public concern.
Key topics like \emph{carbon tax}, \emph{renewable energy}, \emph{sea level} and \emph{greenhouse effect} highlight public interest in policy measures and sustainable solutions. The study also examines mentions and sentiment of US Presidents, revealing nuanced perceptions of their climate policies. Barack Obama and Joe Biden received positive sentiments for proactive climate actions. In contrast, George W. Bush and Donald Trump faced more negative sentiments, particularly post-presidency, highlighting skepticism of their policies. Named Entity Recognition (NER) identified organizations like Google or ExxonMobile as influential in the discourse.
This thesis improves understanding of the evolution of public opinion on climate change and emphasizes the importance of proactive, transparent climate policies. The findings highlight social media's role in shaping public discourse.

\subsubsection{German}
In dieser Arbeit werden Muster der öffentlichen Stimmung und des Diskurses über den Klimawandel in sozialen Medien untersucht, indem Reddit-Daten aus den Jahren 2010 bis 2022 mit Techniken der natürlichen Sprachverarbeitung (NLP) analysiert werden. Die Studie untersucht, wie sich die Häufigkeit und die Stimmung bestimmter Begriffe im Laufe der Zeit entwickeln. Die Ergebnisse zeigen einen signifikanten Anstieg der Diskussionen über den Klimawandel, insbesondere während Veranstaltungen wie den Fridays for Future-Protesten. Die Stimmungsanalyse zeigt, dass die Stimmung insgesamt eher negativ ist, aber in vielen Jahren war die Stimmung positiver und spiegelte den Optimismus in Bezug auf Klimalösungen wider. Die letzten Jahre waren jedoch durch eine zunehmende negative Stimmung gekennzeichnet, was auf eine wachsende Besorgnis der Öffentlichkeit hinweist.
Schlüsselthemen wie \emph{carbon tax}, \emph{renewable energy}, \emph{sea level} und \emph{greenhouse effect} verdeutlichen das öffentliche Interesse an politischen Maßnahmen und nachhaltigen Lösungen. Die Studie untersucht auch die Erwähnungen und Stimmungen der US-Präsidenten und zeigt eine differenzierte Wahrnehmung ihrer Klimapolitik. Barack Obama und Joe Biden wurden für proaktive Klimamaßnahmen positiv bewertet. Im Gegensatz dazu wurden George W. Bush und Donald Trump eher negativ bewertet, insbesondere nach ihrer Präsidentschaft, was die Skepsis gegenüber ihrer Politik verdeutlicht. Named Entity Recognition (NER) identifizierte Organisationen wie Google oder ExxonMobile als einflussreich im Diskurs.
Diese Arbeit verbessert das Verständnis für die Entwicklung der öffentlichen Meinung zum Klimawandel und unterstreicht die Bedeutung einer proaktiven, transparenten Klimapolitik. Die Ergebnisse verdeutlichen die Rolle der sozialen Medien bei der Gestaltung des öffentlichen Diskurses.