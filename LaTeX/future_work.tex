\subsubsection{Limitations}
Several limitations were encountered in this study, which should be considered when interpreting the findings:
\begin{description}
    \item[Data Source Limitation:] The analysis relied only on Reddit data. While Reddit is a popular platform with diverse opinions, it may not fully represent the broader public discourse on climate change. Other social media platforms, news articles, and forums could provide additional perspectives and a more comprehensive understanding.
    \item[Temporal Constraints:] The dataset covers the period from 2010 to 2022. Although this period includes significant developments in climate change discourse, the analysis might miss historical context or earlier trends that could provide deeper insights into the evolution of public sentiment and discourse.
    \item[Sentiment Analysis Accuracy:] Sentiment analysis tools, while powerful, are not perfect. They can misinterpret sarcasm, irony, or context-specific language, leading to potential inaccuracies in sentiment classification. Additionally, the tools used may not account for the nuances of climate change terminology and discussions \cite{liu2012}.
    \item[NER Limitations:] The current Named Entity Recognition (NER) implementation included entity text normalization (e.g., "obama" to "Barack Obama" and "Barack Obama" to "Barack Obama") on a lexicon-based level. Most obvious labels were corrected (e.g., "Trump" was recognized as ORG and corrected to PERSON). However, since this approach is lexicon-based, many incorrect entities remain, and it only covers detected entities. There are likely many occurrences that were not detected at all.
    \item[Language and Regional Bias:] The analysis focused on English-language comments, potentially overlooking significant contributions in other languages. Furthermore, the dataset is US-centric, missing perspectives from other continents. Including data from regions like China (e.g., Weibo) could enhance data quality and allow for analysis of regional differences in events and discourse \cite{8554131}.
    \item[Static Analytical Models:] The analytical models used in this study are static and may not adapt well to evolving language and discourse patterns. Continuous updates and improvements of these models are necessary to maintain accuracy over time.
\end{description}

\subsubsection{Future Work}
Building on the findings and addressing the limitations, several possibilities for future research are suggested:
\begin{description}
    \item[Incorporating Diverse Data Sources:] Future studies should integrate data from multiple social media platforms (e.g., Twitter, Facebook), news websites, and forums to capture a more complete overview of public discourse on climate change. This would help mitigate platform-specific biases and provide a richer dataset.
    \item[Extending Temporal Coverage:] Expanding the dataset to include earlier periods could offer valuable historical context and reveal longer-term trends in climate change discourse. This would help in understanding how public sentiment and discussion topics have evolved over decades.
    \item[Enhancing Sentiment Analysis Techniques:] Employing more advanced sentiment analysis techniques, such as deep learning models and context-aware algorithms, could improve the accuracy of sentiment classification. Integrating domain-specific lexicons and training models on climate-related texts can also improve performance \cite{liu2012}.
    \item[Improving Entity Recognition:] Improving NER models to better capture climate-related entities and integrating manual validation steps can improve the accuracy and completeness of entity recognition. Future work could also explore the relationships between entities to provide deeper insights into the discourse network. Advanced methods like contextualized embeddings (e.g., BERT) could be used to improve entity recognition \cite{Devlin2019BERTPO}
    \item[Language and Regional Bias:] Expanding the analysis to include comments in multiple languages and considering regional variations in discourse can provide a more global perspective on climate change discussions. Cooperation with linguists and regional experts can help in adapting models to different linguistic and cultural contexts. Additionally, adding data from other regions, such as China and platforms like Weibo, can improve the dataset and provide a fuller analysis of regional differences \cite{8554131}.
    \item[Dynamic Models:] Developing dynamic models that can adapt to changing language and discourse patterns over time would improve the strength of the analysis. These models could be periodically updated with new data to keep their relevance and accuracy.
    \item[Longitudinal Studies:] Conducting longitudinal studies to track changes in public sentiment and discourse over extended periods can provide deeper insights into the factors driving these changes. This approach can help in identifying long-term trends and the impact of major events on public opinion.
    \item[Policy Impact Analysis:] Future research could focus on analyzing the impact of specific climate policies and initiatives on public sentiment and discourse. This would involve correlating policy changes with shifts in sentiment and discussion topics to assess their effectiveness and public reception.
\end{description}

By addressing these limitations and exploring these paths for future research, a broader and deep understanding of public discourse on climate change can be achieved, eventually contributing to more effective communication and policy-making in this critical area.