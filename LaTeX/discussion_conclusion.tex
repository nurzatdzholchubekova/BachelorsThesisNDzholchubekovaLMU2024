\section{Summary of Findings}
\subsection{Data Trends and Sentiment Analysis}
The analysis began with examining the frequency and sentiment of climate change discussions on Reddit from 2010 to 2022. The increasing volume of discussions over the years, especially during significant climate events like the Fridays for Future protests and the Australian bushfires in 2019, highlighted how public interest and engagement grew sharply in response to key events. This initial analysis provided a foundation for understanding broader patterns in discourse.

Sentiment analysis revealed that, in absolute terms, there were slightly more negative comments (2,109,360) than positive comments (2,067,883). Neutral comments were significantly fewer (423,455). However, on a yearly basis, it was observed that positive sentiment often exceeded negative sentiment in most years. Specifically, from 2010 to 2017, and in 2021, positive comments were more frequent. Only in the years 2018, 2019, 2020, and 2022 did negative comments dominate positive ones. This indicates that while overall discussions had a slightly negative tone, there were significant periods where positive sentiment was more prevalent, reflecting optimism and proactive engagement. Nonetheless, the last years, particularly 2018, 2019, 2020, and 2022, were characterized by a predominance of negative sentiment, suggesting increased public concern or frustration with climate-related issues during these periods.

\subsection{Key Terms and Entities}
An analysis of specific words and phrases (unigrams and bigrams) identified key terms such as "carbon tax", "renewable energy", "sea level", and "greenhouse effect". This revealed a strong public interest in policy measures and sustainable solutions. Named Entity Recognition (NER) helped identify important organizations and political figures central to the climate change discourse, such as the United Nations.

\subsection{US Presidents and Climate Change}
Mentions of US Presidents from 2010 to 2022 provided insight into public perceptions of their climate policies. Barack Obama and Joe Biden generally received more positive sentiment, reflecting approval of their proactive climate actions. Conversely, George W. Bush and Donald Trump faced more negative sentiment, particularly post-presidency, highlighting skepticism and criticism of their climate policies. Donald Trump's untypically high frequency of mentions, even after his presidency, was attributed to his significant social media presence and polarizing views on climate change.

\section{Connecting the Findings}
The exploration began with a broad analysis of data trends and sentiment, which set the foundation for a more focused investigation into specific terms and entities. This journey brought to light the public's nuanced views on climate change policies and key figures influencing the discourse. By connecting these layers of analysis, the core research question was addressed:


\textbf{What are the patterns of public sentiment and discourse on climate change in social media, and how do they evolve over time?}
\begin{enumerate}
    \item \textbf{How does the frequency of specific unigrams and bigrams about climate change evolve over time?}
    \begin{itemize}
        \item The analysis showed that the frequency of terms like "carbon tax" and "renewable energy" increased over time, reflecting a shift in public interest towards actionable solutions. Early discussions focused on basic concepts like the "greenhouse effect", while later conversations included more specific and actionable topics such as "carbon emissions", "clean energy", and "climate policy". These shifts indicate a maturation in public engagement with climate change, moving from awareness to a focus on policy and action.
    \end{itemize} 
    \item \textbf{How does the sentiment associated with specific unigrams and bigrams about climate change evolve over time?}
    \begin{itemize}
        \item Sentiment analysis revealed a complex picture. While the overall tone of discussions had slightly more negative comments in absolute terms, many years showed a predominance of positive sentiment, particularly from 2010 to 2017 and in 2021. Negative sentiment was more dominant in 2018, 2019, 2020, and 2022. This indicates a fluctuating but generally optimistic public engagement with climate solutions during most of the period studied. Nonetheless, the last years, particularly 2018, 2019, 2020, and 2022, were characterized by a predominance of negative sentiment, suggesting increased public concern or frustration with climate-related issues during these periods.
    \end{itemize}
\end{enumerate}

\section{Integrated Discussion Points}
\subsection{The Role of Social Media and Key Events}
Social media platforms like Reddit have a significant impact on shaping public discourse on climate change. The platform allows for a diverse range of opinions and higher engagement compared to traditional media. This dynamic environment can raise both constructive discussions and polarized debates. Significant events and movements, such as the Paris Agreement and major natural disasters, heavily influence the volume and sentiment of climate discussions. These events often trigger rise in both positive and negative sentiments, reflecting the public's emotional response. For example, the Fridays for Future protests led by Greta Thunberg generated a significant increase in climate change discussions, highlighting the power of social movements in driving public discourse.

\subsection{Public Perception of Policies and Key Figures}
The analysis showed that public perception of climate policies is heavily influenced by their effectiveness and transparency. Proactive and transparent policies tend to receive positive sentiment, while regressive or harmful policies attract criticism. This is evident in the sentiment towards different US Presidents. Barack Obama and Joe Biden received positive sentiments for their proactive climate actions, while George W. Bush and Donald Trump faced negative sentiments due to skepticism and criticism of their policies. The critique of former presidents underscores the importance of sustainable and forward-thinking climate policies that can withstand long-term evaluations. As the impacts of their policies become clearer over time, public sentiment tends to become more critical, highlighting the need for policies that are both effective and adaptable to future challenges.

\subsection{Evolution of Climate Terminology and Public Engagement}
The shift in frequently used terms over time indicates that public understanding and discourse on climate change have evolved. Early discussions focused on basic concepts like the "greenhouse effect", but over time, they have shifted towards more specific and actionable topics such as "carbon tax", "renewable energy", and "sea level rise". This trend suggests that public engagement with climate issues is becoming more sophisticated and solution-oriented. As people become more informed, their discussions reflect a deeper understanding of the complexities of climate change and a greater focus on practical solutions and policies. This evolution in terminology and focus points to an increasing public awareness and readiness to support and promote for actionable climate measures.

\section{Conclusion}
This thesis demonstrates the value of using NLP techniques to analyze large-scale textual data and gain insights into public discourse on climate change. By examining discussions on Reddit, a detailed understanding of how public opinion evolves and reacts to key events and policies has been provided.

The findings emphasize the importance of proactive and transparent climate policies that align with public expectations and scientific recommendations. They also highlight the significant role of social media in shaping public discourse and the need for communicators to balance critical discussions with positive and solution-oriented narratives.

Moving forward, these insights can inform policymakers, environmental organizations, and communicators in their efforts to engage the public more effectively and encourage constructive dialogue on climate change. Future studies should explore other social media platforms and integrate more different datasets to capture a broader range of perspectives and cultural contexts.

Understanding the dynamics of public discourse on climate change is crucial for driving effective action and achieving meaningful progress in addressing this pressing global issue. This thesis contributes to that understanding by providing detailed insights into the patterns and sentiments that characterize climate change discussions on Reddit.