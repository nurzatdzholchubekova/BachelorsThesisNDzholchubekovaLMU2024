This chapter explores the application of natural language processing (NLP) techniques in analyzing discourse surrounding climate change, merging significant findings from notable research, alongside exploring contributions from other foundational works that use NLP to analyze environmental communications comprehensively.

Understanding climate change discourse is crucial as it reflects public perception, influences policy-making, and drives collective action. Climate change is not only a scientific issue but also a deeply social one, involving varied stakeholders including policymakers, scientists, activists, and the general public. The ability of NLP to process and analyze large volumes of text data from diverse sources offers an opportunity to uncover patterns, trends, and sentiments within climate change discussions. NLP can contribute to this field by automating the analysis of vast amounts of textual data, which would be impractical to process manually. It can identify emerging trends, monitor shifts in public sentiment, detect misinformation, and even predict future discourse patterns. Through sentiment analysis, topic modeling, and other NLP techniques, researchers can gain insights into public opinions, identify key themes, and understand the evolution of climate change narratives over time. This information is vital for formulating effective communication strategies, shaping public policies, and promoting greater engagement and action on climate issues.

Sentiment analysis, one of the primary tools in NLP's arsenal, has been crucial in understanding public opinions on climate policies. \cite{Amangeldi} used sentiment analysis to examine reactions on social media platforms, revealing a spectrum of public reactions from strong support to intense opposition towards climate change policies. This study highlights the polarized nature of public sentiment and underscores the challenge in communicating climate policies effectively. Additionally, \cite{PerezFigueroa2024} conducted a content analysis of social media discourse during Hurricane María, noting an increase in negative sentiments when traditional media sources were unable to provide timely coverage. Their research indicates that public sentiment can be significantly influenced by immediate environmental events, suggesting that timely and empathetic communication on social media may help mitigate negative perceptions.

Topic modeling, particularly Latent Dirichlet Allocation (LDA), has been essential in uncovering common themes within climate-related discourse. \cite{Ejaz2023} used Latent Dirichlet Allocation (LDA) to analyze 7,655 climate change-related news articles published between 2010 and 2021 in three leading Pakistani newspapers. This study observed a significant increase in climate change coverage over the years, with a notable shift from mitigation strategies to climate adaptation. This shift reflects a broader change in the climate change narrative from prevention to management, emphasizing adaptability and coping mechanisms in response to the increasing impacts of climate change. 

\cite{foderaro2023argumentative} structured unstructured debate data to highlight how different argumentation styles impacted public opinion and policy-making. Their findings suggest that certain argumentative strategies, particularly those that are clear and emotionally persuasive, are more effective in influencing public opinion. Further expanding on this, \cite{CINDERBY2023100143} investigated online forum discussions to identify which types of arguments are most persuasive in promoting pro-environmental behaviors. They discovered that arguments relating personal health benefits were more effective than those emphasizing long-term environmental benefits, suggesting that personalizing the impacts of climate change might lead to more significant public engagement.

\cite{doe2021climate} undertook a complete analysis of the rhetorical and discursive strategies employed in climate change debates. They collected a vast corpus from varied sources including social media, policy debates, and news articles, applying both qualitative and quantitative NLP techniques. Their findings reveal significant linguistic differences between climate change skeptics and proponents, with each group employing distinctively impactful keywords and phrases to influence specific audiences. This study not only highlights the polarization in climate discourse but also offers valuable strategies for developing more persuasive environmental communications.

These studies collectively underscore the use of NLP in extracting meaningful insights from vast amounts of textual data related to climate change. However, challenges such as data diversity, the complexity of natural language, and the need for interdisciplinary approaches remain widespread. \cite{veritasnlp2024} call for more robust data handling techniques to avoid biases in model training.

This chapter has established a foundation for further explorations of specific NLP techniques and case studies, highlighting the evolving role of artificial intelligence in environmental discourse analysis. My research, focused on a large dataset from Reddit and employing sentiment analysis, aims to deepen the understanding of textual analyses within climate debates, offering actionable insights that can influence public perception and policy formulation. This research promises not only to enrich our understanding of the narrative dynamics but also to provide practical strategies for improving engagement and encouraging constructive dialogue around climate change issues.